% Options for packages loaded elsewhere
\PassOptionsToPackage{unicode}{hyperref}
\PassOptionsToPackage{hyphens}{url}
%
\documentclass[
]{article}
\usepackage{amsmath,amssymb}
\usepackage{iftex}
\ifPDFTeX
  \usepackage[T1]{fontenc}
  \usepackage[utf8]{inputenc}
  \usepackage{textcomp} % provide euro and other symbols
\else % if luatex or xetex
  \usepackage{unicode-math} % this also loads fontspec
  \defaultfontfeatures{Scale=MatchLowercase}
  \defaultfontfeatures[\rmfamily]{Ligatures=TeX,Scale=1}
\fi
\usepackage{lmodern}
\ifPDFTeX\else
  % xetex/luatex font selection
\fi
% Use upquote if available, for straight quotes in verbatim environments
\IfFileExists{upquote.sty}{\usepackage{upquote}}{}
\IfFileExists{microtype.sty}{% use microtype if available
  \usepackage[]{microtype}
  \UseMicrotypeSet[protrusion]{basicmath} % disable protrusion for tt fonts
}{}
\makeatletter
\@ifundefined{KOMAClassName}{% if non-KOMA class
  \IfFileExists{parskip.sty}{%
    \usepackage{parskip}
  }{% else
    \setlength{\parindent}{0pt}
    \setlength{\parskip}{6pt plus 2pt minus 1pt}}
}{% if KOMA class
  \KOMAoptions{parskip=half}}
\makeatother
\usepackage{xcolor}
\usepackage[margin=1in]{geometry}
\usepackage{longtable,booktabs,array}
\usepackage{calc} % for calculating minipage widths
% Correct order of tables after \paragraph or \subparagraph
\usepackage{etoolbox}
\makeatletter
\patchcmd\longtable{\par}{\if@noskipsec\mbox{}\fi\par}{}{}
\makeatother
% Allow footnotes in longtable head/foot
\IfFileExists{footnotehyper.sty}{\usepackage{footnotehyper}}{\usepackage{footnote}}
\makesavenoteenv{longtable}
\usepackage{graphicx}
\makeatletter
\def\maxwidth{\ifdim\Gin@nat@width>\linewidth\linewidth\else\Gin@nat@width\fi}
\def\maxheight{\ifdim\Gin@nat@height>\textheight\textheight\else\Gin@nat@height\fi}
\makeatother
% Scale images if necessary, so that they will not overflow the page
% margins by default, and it is still possible to overwrite the defaults
% using explicit options in \includegraphics[width, height, ...]{}
\setkeys{Gin}{width=\maxwidth,height=\maxheight,keepaspectratio}
% Set default figure placement to htbp
\makeatletter
\def\fps@figure{htbp}
\makeatother
\setlength{\emergencystretch}{3em} % prevent overfull lines
\providecommand{\tightlist}{%
  \setlength{\itemsep}{0pt}\setlength{\parskip}{0pt}}
\setcounter{secnumdepth}{-\maxdimen} % remove section numbering
\ifLuaTeX
  \usepackage{selnolig}  % disable illegal ligatures
\fi
\IfFileExists{bookmark.sty}{\usepackage{bookmark}}{\usepackage{hyperref}}
\IfFileExists{xurl.sty}{\usepackage{xurl}}{} % add URL line breaks if available
\urlstyle{same}
\hypersetup{
  pdftitle={Meu primeiro relatório},
  hidelinks,
  pdfcreator={LaTeX via pandoc}}

\title{Meu primeiro relatório}
\author{}
\date{\vspace{-2.5em}2024-01-18}

\begin{document}
\maketitle

\hypertarget{objetivo-geral}{%
\section{Objetivo geral}\label{objetivo-geral}}

Neste documento vamos apresentar funcionalidades do \emph{R Markdown}, e
depois do \emph{Quarto}, utilizando dados no meio do documento.

Os objetivos específicos desta análise são:

\begin{itemize}
\item
  fazer uma tabela no meio do documento;
\item
  fazer um gráfico no meio do documento.
\end{itemize}

\hypertarget{material-do-documento}{%
\subsection{Material do documento}\label{material-do-documento}}

Neste documento vamos usar a base de dados de filme do IMDB, famoso site
para registro de informações e avaliações sobre filmes. Abaixo temos uma
amostra das informações contidas na base para 1 filme:

\begin{longtable}[]{@{}
  >{\raggedright\arraybackslash}p{(\columnwidth - 38\tabcolsep) * \real{0.0204}}
  >{\raggedright\arraybackslash}p{(\columnwidth - 38\tabcolsep) * \real{0.0184}}
  >{\raggedleft\arraybackslash}p{(\columnwidth - 38\tabcolsep) * \real{0.0102}}
  >{\raggedright\arraybackslash}p{(\columnwidth - 38\tabcolsep) * \real{0.0327}}
  >{\raggedright\arraybackslash}p{(\columnwidth - 38\tabcolsep) * \real{0.0347}}
  >{\raggedleft\arraybackslash}p{(\columnwidth - 38\tabcolsep) * \real{0.0163}}
  >{\raggedright\arraybackslash}p{(\columnwidth - 38\tabcolsep) * \real{0.0102}}
  >{\raggedright\arraybackslash}p{(\columnwidth - 38\tabcolsep) * \real{0.0163}}
  >{\raggedleft\arraybackslash}p{(\columnwidth - 38\tabcolsep) * \real{0.0204}}
  >{\raggedleft\arraybackslash}p{(\columnwidth - 38\tabcolsep) * \real{0.0163}}
  >{\raggedleft\arraybackslash}p{(\columnwidth - 38\tabcolsep) * \real{0.0245}}
  >{\raggedleft\arraybackslash}p{(\columnwidth - 38\tabcolsep) * \real{0.0204}}
  >{\raggedleft\arraybackslash}p{(\columnwidth - 38\tabcolsep) * \real{0.0306}}
  >{\raggedright\arraybackslash}p{(\columnwidth - 38\tabcolsep) * \real{0.0245}}
  >{\raggedright\arraybackslash}p{(\columnwidth - 38\tabcolsep) * \real{0.0531}}
  >{\raggedright\arraybackslash}p{(\columnwidth - 38\tabcolsep) * \real{0.0388}}
  >{\raggedright\arraybackslash}p{(\columnwidth - 38\tabcolsep) * \real{0.2347}}
  >{\raggedright\arraybackslash}p{(\columnwidth - 38\tabcolsep) * \real{0.2918}}
  >{\raggedleft\arraybackslash}p{(\columnwidth - 38\tabcolsep) * \real{0.0429}}
  >{\raggedleft\arraybackslash}p{(\columnwidth - 38\tabcolsep) * \real{0.0429}}@{}}
\toprule\noalign{}
\begin{minipage}[b]{\linewidth}\raggedright
id\_filme
\end{minipage} & \begin{minipage}[b]{\linewidth}\raggedright
titulo
\end{minipage} & \begin{minipage}[b]{\linewidth}\raggedleft
ano
\end{minipage} & \begin{minipage}[b]{\linewidth}\raggedright
data\_lancamento
\end{minipage} & \begin{minipage}[b]{\linewidth}\raggedright
generos
\end{minipage} & \begin{minipage}[b]{\linewidth}\raggedleft
duracao
\end{minipage} & \begin{minipage}[b]{\linewidth}\raggedright
pais
\end{minipage} & \begin{minipage}[b]{\linewidth}\raggedright
idioma
\end{minipage} & \begin{minipage}[b]{\linewidth}\raggedleft
orcamento
\end{minipage} & \begin{minipage}[b]{\linewidth}\raggedleft
receita
\end{minipage} & \begin{minipage}[b]{\linewidth}\raggedleft
receita\_eua
\end{minipage} & \begin{minipage}[b]{\linewidth}\raggedleft
nota\_imdb
\end{minipage} & \begin{minipage}[b]{\linewidth}\raggedleft
num\_avaliacoes
\end{minipage} & \begin{minipage}[b]{\linewidth}\raggedright
direcao
\end{minipage} & \begin{minipage}[b]{\linewidth}\raggedright
roteiro
\end{minipage} & \begin{minipage}[b]{\linewidth}\raggedright
producao
\end{minipage} & \begin{minipage}[b]{\linewidth}\raggedright
elenco
\end{minipage} & \begin{minipage}[b]{\linewidth}\raggedright
descricao
\end{minipage} & \begin{minipage}[b]{\linewidth}\raggedleft
num\_criticas\_publico
\end{minipage} & \begin{minipage}[b]{\linewidth}\raggedleft
num\_criticas\_critica
\end{minipage} \\
\midrule\noalign{}
\endhead
\bottomrule\noalign{}
\endlastfoot
tt0023352 & Prestige & 1931 & 1932-01-22 & Adventure, Drama & 71 & USA &
English & NA & NA & NA & 5.7 & 240 & Tay Garnett & Harry Hervey, Tay
Garnett & RKO Pathé Pictures & Ann Harding, Adolphe Menjou, Melvyn
Douglas, Ian Maclaren, Guy Bates Post, Rollo Lloyd, Clarence Muse, Tetsu
Komai & A woman travels to a French penal colony in Indo China to be
with her fiancée, the commander. She arrives to find that he is now an
alcoholic. & 12 & 2 \\
\end{longtable}

\hypertarget{evoluuxe7uxe3o-das-notas-dos-filmes-no-tempo}{%
\subsection{Evolução das notas dos filmes no
tempo}\label{evoluuxe7uxe3o-das-notas-dos-filmes-no-tempo}}

Abaixo registramos a relação entre a variável tempo e nota no site:

\includegraphics{primeiro_exemplo_rmd_files/figure-latex/unnamed-chunk-2-1.pdf}

\hypertarget{outras-caracteruxedsticas-do-texto}{%
\subsection{Outras características do
texto}\label{outras-caracteruxedsticas-do-texto}}

\textbf{Negrito}

\begin{itemize}
\tightlist
\item
  Ponto de parágrafo
\end{itemize}

Lista de itens:

\begin{itemize}
\tightlist
\item
  item 1
\item
  item 2
\item
  item 3
\end{itemize}

Lista de itens numerada:

\begin{enumerate}
\def\labelenumi{\arabic{enumi}.}
\tightlist
\item
  primeiro
\item
  segundo
\item
  terceiro
\end{enumerate}

ele conta sozinho tb:

\begin{enumerate}
\def\labelenumi{\arabic{enumi}.}
\tightlist
\item
  a
\item
  b
\item
  c
\end{enumerate}

eu uso pra colorir a letra desse jeito no Md no Python: {Este texto está
em vermelho.}

\end{document}
